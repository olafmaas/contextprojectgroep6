\documentclass[a4paper, fleqn]{article}
% mathematische symbolen
\usepackage{geometry}
\usepackage{enumerate}
\usepackage{amsmath}
\setcounter{MaxMatrixCols}{20}
\geometry{verbose,a4paper,tmargin=2.5cm,bmargin=2.5cm,lmargin=2.5cm,rmargin=2.5cm}
\usepackage[dutch]{babel}



\title{$^2$Buttons $^x$Moles} 
\author{Olaf Maas}

\begin{document}
\maketitle
\section{Omschrijving Spel}
De groep van wachtende mensen wordt opgesplitst in twee groepen. In de samenstelling van deze groepen gaan zij de strijd aan in verschillende spellen. Het hoofdbord bevind zich op 1 groot scherm. Op hun mobiel heeft elke speler de beschikking over twee knoppen. 

Naast de normale spelers bevat elke team een aantal saboteurs, deze saboteurs mogen het andere team saboteren. De saboteurs hebben een lijst van acties waaruit ze kunnen kiezen, echter kosten deze acties wel punten uit de totale pot. De saboteurs worden elke ronde gewisseld. Voorbeelden van acties van de saboteur zijn: be\"invloeding stemmen, omdraaien knoppen, ingewikkelder maken spellen en spelers blokkeren. 

Het spel wordt eens per periode gereset, de teamindeling wordt gedaan door de server, in principe zijn de teams in balans.  Af en toe is het ook mogelijk voor een team om te beslissen of ze gaan gokken met hun punten. Zij kunnen dan gezamenlijk beslissen of ze geld inzetten en hoeveel. Uiteraard kan men 

\section{Idee\"en voor minigames}
\begin{enumerate}
\item Kopi\"eren van acties die op het scherm worden getoond. (A, B, niks, beide).
\item Buttonbash spel.
\item Patronen onthouden: ABAABBBA
\item Doolhof, democratisch beslissen kant A of B.
\item Democratisch Blackjack
\item A/B quiz: spreekt voor zich. 
\item Beurs: 1 team bepaald de koers door te stemmen of de koers omhoog of omlaag moet, het andere team koopt/verkoopt aandelen. Het team dat de koer bepaald moet unaniem stemmen om de koers ook daadwerkelijk omlaag of omhoog te krijgen, anders gebeurt het tegenovergestelde. 
\end{enumerate}
\end{document}