\documentclass[a4paper, fleqn]{article}
% mathematische symbolen
\usepackage{geometry}
\usepackage{enumerate}
\usepackage{amsmath}
\setcounter{MaxMatrixCols}{20}
\geometry{verbose,a4paper,tmargin=2.5cm,bmargin=2.5cm,lmargin=2.5cm,rmargin=2.5cm}
\usepackage[dutch]{babel}



\title{Collaborative Pictionary} 

\begin{document}
\maketitle
\section{Omschrijving Spel}
De groep wachtende mensen wordt in twee groepen gesplitst. Groep 1 moet gezamenlijk een object tekenen en de andere groep moet dit object raden. Het doel van het spel is zoveel mogelijk punten te vergaren. 

Elke speler heeft op het scherm van zijn smartphone of een interface waarmee hij woorden kan raden. Of een klein deel van het grote scherm waarop hij kan tekenen. Als een speler het woord heeft geraden krijgt hij daarvoor punten. 

Dit spel is geschikt voor een plek waar een grote ruimte is waar iedereen zicht heeft op het hoofdbeeldscherm. Uiteraard kan er ook met meerdere beamers worden gewerkt. Mensen moeten wel enige tijd hebben om te wachten. 

\section{Benodigdheden}
\begin{enumerate}
\item Beeldscherm/Beamer
\item Spelers met een smartphone
\item Server
\end{enumerate}


\end{document}